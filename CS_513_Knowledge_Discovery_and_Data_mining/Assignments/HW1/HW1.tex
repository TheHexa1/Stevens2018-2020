\documentclass{exam}

\usepackage{amsmath}

\usepackage{amssymb}

\usepackage{graphicx}

\usepackage{cite}
\usepackage{color}
\usepackage{float} 
\usepackage{setspace}
\usepackage{hyperref}
\usepackage[linewidth=1pt]{mdframed}
\usepackage{tcolorbox}
\usepackage{hyperref}
\newcommand{\xx}{{\bf{x}}}
\newcommand{\yy}{{\bf{y}}}
\newcommand{\ww}{{\bf{w}}}
\newcommand{\uu}{{\bf{u}}}

\pagestyle{headandfoot}
\runningheadrule
\firstpageheader{CS513}{Knowledge Discovery and Data mining}{Viveksinh Solanki - 10441787}



\begin{document}

\thispagestyle{headandfoot}
\rhead{Viveksinh Solanki - 10441787}

\section{Homework}
\subsection {}
Given, \\ \\
$P(J) = 0.2 \quad \quad \quad P(S) = 0.3 \quad \quad \quad P(J \cap S) =  0.08$ \\

a) Susan was at the bank last Monday. What’s the probability that Jerry was there too?
\begin{align}
P(J | S) = \frac{P(J \cap S)}{P(S)} = \frac{0.08}{0.3} = 0.27
\end{align}

b) Last Friday, Susan wasn’t at the bank. What’s the probability that Jerry was there?
\begin{align}
P(J | S') = \frac{P(J \cap S')}{P(S)} = \frac{P(J \cap S)-P(S)}{1-P(S)} = \frac{0.12}{0.7} = 0.17
\end{align}

c) Last Wednesday at least one of them was at the bank. What is the probability that both of them were there?
\begin{align}
\frac{P(J \cap S)}{P(J \cup S)} &= \frac{0.08}{0.42} = 0.1904
\end{align}

\newpage

\subsection {}
Given, \\ \\
$P(H) = 0.8 \quad \quad \quad P(S) = 0.9 \quad \quad \quad P(H \cup S) =  0.91$ \\

a) What is the probability that only Harold gets a “B”?
\begin{align}
P(H \cup S) - P(S)= 0.91 - 0.9 = 0.01
\end{align}

b) What is the probability that only Sharon gets a “B”?
\begin{align}
P(H \cup S) - P(H)= 0.91 - 0.8 = 0.11
\end{align}

c) What is the probability that both won’t get a “B”?
\begin{align}
P(J \cup S)' = 1 - P(H \cup S) = 1 - 0.91 = 0.09
\end{align}

\newpage

\subsection{}
a) Are the events “Jerry is at the bank” and “Susan is at the bank” independent? \\ 
No

\subsection{}
a) Are the events “the sum is 6” and “the second die shows 5” independent? \\ 
No \\ \\ 
b) Are the events “the sum is 7” and “the first die shows 5” independent? \\ 
Yes 

\subsection {}
Given, \\ \\
$P(TX) = 0.6 \quad \quad \quad P(NJ) = 0.9 \quad \quad \quad P(AK) = 1 - P(TX) - P(NJ) =  0.3$ \\
$P(oil | TX) = 0.3 \quad \quad \quad P(oil | NJ) = 0.1 \quad \quad \quad P(oil | AK) = 0.2$ \\

a) What’s the probability of finding oil?
\begin{align}
P(oil) &= \sum_{s}P(oil, state) \\
         &= \sum P(oil | state)P(state) \\
         &= P(oil | TX) P(TX) + P(oil | NJ) P(NJ) + P(oil | AK) P(AK) \\
         &= 0.3*0.6 + 0.2*0.3 + 0.1*0.1 \\
         &= 0.18 + 0.06 + 0.01 \\
         &= 0.25
\end{align}

b) The company decided to drill and found oil. What is the probability that they drilled in TX?
\begin{align}
P(TX | oil) &=  \frac{P(oil | TX)P(TX)}{P(oil)} \\
	      &= \frac{0.18}{0.25} \\
	      &= 0.72
\end{align}

\newpage

\subsection{}
$From\, given\, table,$ \\ 

a) What is the probability that a passenger did not survive?
\begin{align}
P(not\, survived) = \frac{1490}{2201} = 0.68
\end{align} 

b) What is the probability that a passenger was staying in the first class?
\begin{align}
P(cabin=1st) = \frac{325}{2201} = 0.15
\end{align}

c) Given that a passenger survived, what is the probability that the passenger was staying in the first class?
\begin{align}
P(cabin=1st | survived) = \frac{203}{711} = 0.29
\end{align}

d) Are survival and staying in the first class independent? \\
No

e) Given that a passenger survived, what is the probability that the passenger was staying in the first class and the passenger was a child?
\begin{align}
P(cabin=1st \cap age=child | survived) = \frac{6}{711} = 0.008
\end{align}

f) Given that a passenger survived, what is the probability that the passenger was an adult?
\begin{align}
P(age = adult| survived) = \frac{654}{711} = 0.92
\end{align}

g) Given that a passenger survived, are age and staying in the first class independent? \\
No

\newpage
\subsection{}

As age and cabin class are independent,
\begin{align}
P(Age=adult, Cabin=1st) &= P(Age=adult) * P(Cabin=1st) \\
&= \frac{2092}{2201} *  \frac{325}{2201} \\
&= 0.1403 
\end{align}

Hence, \# of adult passengers who were in 1st class would be 
\begin{align}
&= 0.1403* Total Passengers \\
&= 0.1403*2201\\
&\propto 309
\end{align}

In a similar way, we can calculate all missing values in Total passengers' table as well as survived/not survived tables.
Calculated values are given in respective tables.

\begin{table}[H]
\caption{Total}
\begin{center}
  \begin{tabular}{c | c | c | c | c | c  }
    \hline
      & 1st & 2nd & 3rd & Crew & Grand Total \\ 
     \hline              
     Adult & 309 & 271 & 671 & 841 & 2092 \\ \hline
     Child & 16 & 14 & 35 & 44 & 109 \\ \hline
     Grand Total & 325 & 285 & 706 & 885 & 2201\\ 
    
    \hline
  \end{tabular}
\end{center}
\end{table}

\begin{table}[H]
  \caption{Survived}
\begin{center}
  \begin{tabular}{c | c | c | c | c | c  }
    \hline
      & 1st & 2nd & 3rd & Crew & Sub Total \\ 
     \hline              
     Adult & 187 & 108 & 164 & 195 & 654 \\ \hline
     Child & 16 & 10 & 14 & 17 & 57\\ \hline
     Sub Total & 203 & 118 & 178 & 212 & 711\\ 
    
    \hline
  \end{tabular}
\end{center}
\end{table}

\begin{table}[H]
\caption{Not Survived}
\begin{center}
  \begin{tabular}{c | c | c | c | c | c  }
    \hline
      & 1st & 2nd & 3rd & Crew & Sub Total \\ 
     \hline              
     Adult & 118 & 161 & 510 & 649 & 1438 \\ \hline
     Child & 4 & 6 & 18 & 24 & 52 \\ \hline
     Sub Total & 122 & 167 & 528 & 673 & 1490 \\ 
    
    \hline
  \end{tabular}
\end{center}
\end{table}


\end{document}